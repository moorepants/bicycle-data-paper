\documentclass{article}

\usepackage{todo}

\title{Kinetic and Kinematic Measurements from an Instrumented Bicycle during
  different maneuevers on and off the treadmill}

\begin{document}

\section{Introduction}

We developed an instrumented bicycle with a rigidified rider which is capable
of accurately measuring a variety of kinematic and kinetic time varying data.
We then performed a large number of experiments (maneuvers) with several
different riders on a treadmill and on indoor level flat ground. The impetus
for most of the experiments was to collect data that was ideal for system and
parameter identification purposes of the bicycle/rider system. The majority of
the data are trials in which the bicycle/rider system was laterally perturbed
by externally applied pulses while the rider attempted to keep either the
bicycle's heading or the bicycle's front wheel tracking a line.

There are few instances of similar data collected for the bicycle/rider system
over history. In the early 70's, a comprehensive study on bicycle dynamics and
control was implemented at CALSPAN where kinematic data was collected from
riderless and ridden bicycles. This data was collected on stripcharts and video
tape. Almost simultaneously, van Lunteren and Stassen developed a bicycle
simulator and collected digital kinematic data of rider's controlling the
simulator. In the 80's Dolye developed an instrumented bicycle that could
measure kinematics digitally. He performed blindfolded slow riding. More
recently, Kooijman developed an instrumented riderless bicycle with kinematic
sensors performing straight running experiments on and off the treadmill while
being perturbed by an unmeasured perturbation. Moore et. al collected kinematic
data of a bicycle. Besides Moore2012 none of these datasets are publicly
available.

% TODO : Should I cite all data collected on bicycles?: de Lorenzo, Cheng,
% Cain, etc? Or just ones intended for system id purposes?

Here we present the first publicly available and organized dataset of a rider
controlling a bicycle with the most important measurements needed for system or
parameter identification in addition to the methods and equipment design and
basic software to process the data. We give both the ``raw data'' and a set of
processed data.

\section{Methods}

\subsection{Instrumentation}

We developed a powered instrumented bicycle to collect the desired data. The
instrumented bicycle includes sensors for measuring both kinematic and kinetic
data and utilized a harness to limit and minimize the rider's motion relative
to the rear frame of the bicycle.

The bicycle had an adjustable torso harness which fixed the rider's torso to
the rear frame of the bicycle. In addition, the rider's feet were place on foot
pegs and their knee's were affixed to the rear frame via strong magnets. Thus
the only significant biomechanical motion is the rider's arm motion for
steering and motion of the head.

The body fixed angular rate and a body fixed point acceleration of the rear
frame were measured with a VN-100 (VectorNav, Austin, TX, USA) inertial
measurement unit. The VN-100 was mounted to the rear frame with its factory X
axis aligned with $\hat{c}_1$, Y axis aligned with the $-\hat{c}_3$, and Z axis
aligned with $\hat{c}_2$ as described in Moore2012. I made use of the VN-100's
ability to output it's measurements with respect to a different reference frame
than the factory frame and aligned the X, Y and Z axes with the $\hat{c}_1$,
$\hat{c}_2$ and $\hat{c}_3$ axes, respectively.  This pre-output rotation
matrix was recorded in the meta data for each run. The steer rate gyro was
attached such that its axis was aligned with $\hat{e}_3$.

\todo{Need to have the location of the VN-100 point}

The roll angle of the rear frame relative to level ground was measured with a
custom built lightweight trailer in which a potentiometer output voltage
proportional to the roll angle.

The body fixed angular rate about the steer axis of the bicycle front frame was
measured with a single rate gyro. The relative steer angle between the front
and rear frames was measured with a potentiometer.

The relative rate of the rear wheel with respect to the rear frame was measured
with a electric generator which produced a voltage proportional to the rate.

Most trials included a lateral perturbation force. This force was applied just
below the seat via a rod (or rope) that was held normal to the rear frame
plane. The load was measured with a linear 100 lb load cell (SSM-100,
Interface).

A 150 in-lb torque sensor (TFF350, Futek) was mounted inline with the steer
axis inbetween the top of the fork steer tube and the handlebar stem. It was
mounted between two sets of headset bearings and isolated from loads other than
pure axial torques via a zero backlash telescoping double universal joint.

The potentiometers, generator, and linear load cell required manual
calibrations, ideal before each set of trials. The manufacturers of the
remaining sensors included scaling factors.

The VN-100 signals were recorded independently from the other sensors which
were collected through an analog data acquisition unit (USB-6218, National
Instruments). Thus there are variable time delays between the sets of signals.
To find the time delay we attached an additional three axis accelerometer to
the VN-100 and sampled it through the USB-6218. At the initiation of each trial
the bicyclist has to traverse a bump which give a unique vertical acceleration
pattern. The time delay can be found by computing the autocorrelation of two
signals and applied to the other signals appropriately.

The data was collected from the VN-100 development board and the NI USB-6218
via a netbook laptop (Eee PC 1001P-MU17-WT, ASUSTek Computer Inc., Tapei,
Taiwan) with a Matlab application, BicycleDAQ.

\subsection{Experiments}

Our main experimental design was focused around a reasonable way to excite the
rider/bicycle system to create data rich enough to be used in system and
parameter identification techniques.

We performed the experiments in two different environments: on a treadmill and
in a large indoor gymnasium. The horse treadmill (Graber Ag Kagra Mustang 2200)
is 1 meter wider and 5 meters long and has a speed range from 0.5 m/s to 17
m/s. The gymnasium floor was made of a stiff rubber and provided a rectangular,
wind-free space of about 100' by 180' (30 m by 55 m). We rode around the
perimeter to build up speed and did our maneuvers on a straight section about
100 feet (30 m) long. We were not able to travel at speeds higher than about 7
m/s.

The following list details the meaning of the maneuver labels in the dataset.

\begin{description}
  \item[System Test]
     This is a generic label for data collected during various system tests
     that should not be used for general analysis. This was primarily used to
     check that all sensors were working before each set of experiments.
  \item[Balance] The rider was instructed to simply balance the bicycle and keep
    a relatively constant heading. They were instructed to focus on a point of
    their choosing in the far distance. There was an open door in front of the
    treadmill which allowed the rider to look to a point outside across the
    street. In the pavilion, the rider looked into the rafters of the building
    or at the furthest wall. We had a line taped to the pavilion floor during
    these runs that was still in the periphery of the rider's vision.
  \item[Balance With Disturbance] Same as 'Balance' except that a lateral force
    perturbation is applied. The rider wore a face shield on the side of the
    perturber so no visual cues were available to predict the perturbation time
    or direction. On the treadmill, we sample for 60 to 90 seconds with five to
    eleven perturbations per run. On the pavilion floor we were able to apply
    only two to four perturbations per run due to the length of the track. In
    the early runs (< 204), the lateral force was applied only in the negative
    direction (to the left) and the perturber decided when to apply the
    perturbations. For the later runs (> 203), we applied a random sequence of
    positive and negative perturbations that was unknown to the rider. On the
    treadmill, the rider signaled when they felt stable and the perturbation
    was applied at a random time between 0 and 1 second based on a simple
    computer program. On the pavilion floor, we simply applied the
    perturbations randomly but as soon as the rider felt stable so that we
    could get in as many as possible during each run.
  \item[Track Straight Line] The rider was instructed to focus on a straight
    line that was marked on the ground and he attempted to keep the front wheel
    on the line. The line of sight from the rider's eyes to the line on the
    ground was essentially tangent the top of the front wheel. In the pavilion,
    the line could be seen up to 100 feet ahead, so there was greater
    peripheral view of the line. On the treadmill, there was from 0.5 to 1.5
    meters of preview line available.
  \item[Track Straight Line With Disturbance] Same as "Track Straight Line"
    except that a lateral perturbation force. This was done in the same fashion
    as described in "Balance With Disturbance".
  \item[Lane Change] The rider attempted to track a line in similar fashion as
    the "Track Straight Line" maneuver except that the line was a single lane
    change. On the pavilion floor, the line was taped on the ground and the
    rider was instructed to do whatever felt best to stay on the line Figure X.
    They could use full preview looking ahead, focus on the front wheel and
    line, or a combination of both.  We also tried some lane changes on the
    treadmill but the lack of preview of the line made it especially difficult.
    We were able to manage it by marking a count down on the belt so that the
    rider new when the lane change would arrive. The rider also knew the
    direction of lane change beforehand for all the scenarios.
  \item[Blind With Disturbance] We did a run or two for each rider on the
    pavilion floor with the rider's eyes closed to attempt to completely open
    the heading loop. In hindsight, blind tests would be preferable when
    identifying the rider control system so that only inner roll stabilization
    loop need be analyzed.
  \item[Static Calibration] We took a short duration sample of the sensors'
    signals while no rider was on the bicycle and the bicycle was fixed as
    close as possible to vertical in roll before each set of runs. The static
    accelerometer readings could theoretically give the roll and pitch angles
    of the bicycle frame and be used to account for the bias in the roll angle
    measurements.
\end{description}

\todo{Add the lane change diagram}

The primary set of experimental data was collected on seven different days. The
first few days were mostly trials to test the equipment, procedures and
different maneuvers. The data from the following days is suitable for analysis.

\begin{description}
  \item[February 4 2011 Runs 103-109] These were the first trials on the
    treadmill for preliminary testing. Jason was the only rider. We performed
    lateral deviation tracking with disturbances. The bike fell over, broke and
    we had to cut it short.
  \item[February 28, 2011 Run 115-170] These were the first trials in the
    pavilion. Jason was the only rider. We tried lane changes (115-139),
    lateral deviation tracking with disturbances (140-157), and a mixture of
    heading tracking and lateral deviation tracking with no disturbances
    (158-170). I noted that the slip clutch backlash seemed to be larger than
    the previous day with a guess of about 1 degree.
  \item[March 9, 2011 Runs 180-204] This was the second go at the treadmill,
    still just testing things. Jason was the only rider. We did heading and
    lateral deviation tracking with disturbances and some lane changes. The
    lane changes were 0.25 m wide left and right maneuvers back and forth among
    two lines on the treadmill at 2 m long segments. Countdown markers to give
    an idea when the lane change started were necessary due to the rider's
    limited preview distance. We did the highest speed during any subsequent
    trials at 9 m/s. The 9 m/s runs generated a large amount of noise in the
    lateral force channel. The treadmill elevation was set at 0.1\% upwards
    incline (because it was stuck).
  \item[August 30, 2011 Runs 235-291] Jason and Luke rode and performed heading
    and lateral deviation tasks with and without perturbations at three speeds
    on the treadmill.
  \item[September 6, 2011 Runs 295-318] Charlie performed heading and lateral
    deviation tasks with and without perturbations on the treadmill.
  \item[September 9, 2011 Runs 325-536] Luke, Charlie and Jason performed
    heading and lateral deviation tracking tasks on the Pavilion floor with and
    without perturbations. Most of Luke and Charlie's runs were corrupt due to
    the time synchronization issues.
  \item[September 21, 2011 Runs 538-706] Luke and Charlie repeated the runs
    from September 9th. And we added a couple of blind runs for each of them.
\end{description}

\section{Raw Data}

\subsection{Trial Data}

The software, BicycleDAQ, outputs a single Matlab \verb|.mat| file for each
trial. The \verb|.mat| file contains an array of time series with the raw
voltage readings from the sensors connected to the NI USB-6218, an array of
time series of the rate gyro, accelerometer, and magnetometer readings in
their corresponding SI units from the VN-100 IMU, and it contains meta data
collected from the GUI.

\subsection{Calibration Data}

BicycleDAQ also includes a tool for calibrating various sensors attached to the
instrumented bicycle. This tool outputs a single Matlab \verb|.mat| file for
each sensor calibration. The calibrations were performed before each day of
experimentation.

\end{document}
