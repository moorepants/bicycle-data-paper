\documentclass{article}

\title{Kinetic and Kinematic Measurements from an Instrumented Bicycle during
  different maneuevers on and off the treadmill}

\begin{document}

\section{Introduction}

We developed an instrumented bicycle with a rigidified rider which is capable
of accurately measuring a variety of kinematic and kinetic time varying data.
We then performed a large number of experiments (maneuvers) with several
different riders on a treadmill and on indoor level flat ground. The impetus
for most of the expriements was to collect data that was ideal for system and
parameter identification purposes of the bicycle/rider system. The majority of
the data are trials in which the bicycle/rider system was laterally perturbed
by externally applied pulses while the rider attempted to keep either the
bicycle's heading or the bicycle's front wheel tracking a line.

There are few instances of similar data collected for the bicycle/rider system
over history. In the early 70's, a comprehensive study on bicycle dynamics and
control was implemented at CALSPAN where kinematic data was collected from
riderless and ridden bicycles. This data was collected on stripcharts and video
tape. Almost simulateneouly, van Lunteren and Stassen developed a bicycle
simulator and collected digital kinematic data of rider's controlling the
simulator. In the 80's Dolye developed an instrumented bicycle that could
measure kinematics digitally. He performed blindfolded slow riding. More
recently, Kooijman developed an instrumented riderless bicycle with kinematic
sensors performing straight running expriements on and off the treadmill while
being perturbed by an unmeasured perturbation. Moore et. al collected kinematic
data of a bicycle. Besides Moore2012 none of these datasets are publically
available.

% TODO : Should I cite all data collected on bicycles?: de Lorenzo, Cheng,
% Cain, etc? Or just ones intended for system id purposes?

Here we present the first publically available and organized dataset of a rider
controlling a bicycle with the most important measurements needed for system or
parameter identification in addition to the methods and equipment desing and
basic software to process the data. We give both the ``raw data'' and a set of
processed data.

\section{Raw Data}

\subsection{Trial Data}

The software, BicycleDAQ, outputs a single Matlab \verb|.mat| file for each
trial. The \verb|.mat| file contains an array of time series with the raw
voltage readings from the sensors connected to the NI USB-6218, an array of
time series of the rate gyro, accelerometer, and magnetormeter readings in
their corresponding SI units from the VN-100 IMU, and it contains meta data
collected from the GUI.

\subsection{Calibration Data}

BicycleDAQ also includes a tool for calibrating various sensors attached to the
instrumented bicycle. This tool outputs a single Matlab \verb|.mat| file for
each sensor calibration. The calibrations were performed before each day of
experimentation.

\end{document}
